\section{Topologie}
\subsection{Grundbegriffe}
Sei $(M,d)$ ein metrischer Raum $\wedge \  A \subseteq M$,$a \in M$ dann ist:
\begin{itemize}
    \item  Inneres $\overset{\circ}{A} :=\{a\in M\ \vert \ \exists \ \varepsilon >0         :U_{\varepsilon}(a) \subseteq A\}$
    \item Abschluss $\overline{A}:=\{ a \in M\ \vert \ \forall \varepsilon>0 :               U_\varepsilon(a) \cap A \neq \varnothing \}$
    \item Häufungspunkt $A^\prime:=\{\forall \varepsilon>0 : \.{U}_\varepsilon(a) \cap A         \neq \varnothing \}$ 
    \item Randpunkte $\partial A:= \{a \in M\ \vert \ \forall \varepsilon>0\ \exists         x,y:x\in A , y\in A^c\ \wedge \ x,y \in U_\varepsilon(a) \}$
    \item Bedingung für Isolierte Punkte $a\in A:\ \ \exists \ \varepsilon>0 :               U_\varepsilon(a) \cap A =\{a\}$
    \item \underline{offene Menge}
    \subitem $\iff$ jedes $a\in A$ ist innerer Punkt von $A$ 
    \item \underline{abgeschlossene Menge}
    \subitem $\iff \ A^c$ ist offen 
    \subitem $\iff \partial A \subseteq A$ 
    \subitem $\iff A=\overline{A}$
    \item $K \subset \mathbb{R}$ kompakt $\iff K$ beschränkt und abgeschlossen
\end{itemize}

\subsection{Metriken}
Sei $X$ eine beliebige Menge, dann ist $d: X\times X \rightarrow \mathbb{R}$ eine Metrik wenn für $x,y,z\in X$ folgende Axiome erfüllt sind:
\begin{itemize}
    \item [(M$1$)] Pos. Definitheit $d(x,y)\ge 0 \ \wedge \ d(x,y)=0 \iff x=y$
    \item [(M$2$)] $d(x,y)=d(y,x)$
    \item [(M$3$)] Dreiecksungleichung $d(x,y)\le d(x,z)+d(z,y)$
\end{itemize}

\section{Funktionen und Abbildungen}


\section{Differenzierbarkeit}
\subsection{Grundbegriffe}
\begin{itemize}
    \item partielle Ableitung:\\ $\frac{\partial f}{\partial x_j}(x_0) = d_j f(x_0) = \lim_{h\to 0}\frac{f(x_1^{(0)} \dots x_j^{(0)}+h \dots x_n^{(0)})-f(x_0)}{h}$\\
    f ist partiell diff'bar in $x_0$ falls alle partiellen Ableitungen existieren mit $\nabla f(x_0)$
    \item Stetigkeit in Achsenrichtung: \\
    Ist f in $x_0$ nach $x_i$ diff'bar, so ist f in $x_0$ in Richtung der $x_i$-Achse stetig. ($\neq$ Stetigkeit in alle Richtungen)
    \item Richtungsableitungen:\\
    $e \in \mathbb{R}^n$ ein Richtungsvektor mit $\norm{e}=1$, $x_0 \in U$ existiert \\
    $\lim_{h\to 0}\frac{f(x_0 + he)-f(x_0)}{h}$, so ist dies die Richtungsableitung von f nach $e$ in $x_0$: $\frac{\partial f}{\partial e}(x_0)$\\
    Der Gradient ist ein Spezialfall einer Richtungsableitung und steht immer senkrecht auf den Höhenlinien ($f(x) = c$) und zeigt in die Richtung des stärksten Anstiegs.
\end{itemize}
\subsubsection{Totale Ableitung}
Eine Funktion $f: U\to \mathbb{R}^n$ ist in $x_0 \in U$ total diff'bar, falls f in $x_0$ linearisierbar ist: \\
\begin{equation}
    \bigvee x \in U \  \exists A: f(x) = f(x_0) + A(x-x_0) + r(x,x_0)
\end{equation}
wobei $A$ eine Matrix und $r(x,x_0)$ ein Restterm ist.\\
Bemerkung: $A$ ist die Jacobi-Matrix, bzw bei einer skalarwertigen Funktion der (transponierte) Gradient der Funktion an der entsprechenden Stelle.\\
Ist f in U partiell diff'bar und die partiellen Ableitungen in $x_0$ stetig, dann ist f total diff'bar.\\
Notation: $f'(x_0) = Df(x_0)$\\
Für den Restterm muss gelten 
\begin{equation}
    \lim_{x\to x_0}\frac{\|r(x,x_0)\|}{\|x-x_0\|} = 0
\end{equation}

\subsubsection{Totale Diff'barkeit und Richtungsableitungen}
Ist f in $x_0$ total diff'bar, dann ist f in $x_0$ in jede Richtung diff'bar und für die Richtungsableitungen gilt 
\begin{equation}
    \frac{\partial f}{\partial e} = \langle \nabla f(x_0),e\rangle = - \frac{\partial f}{\partial (-e)} = \lim_{h\to 0}\frac{f(x_0+he)-f(x_0)}{h}
\end{equation}
Beachte, aus Diff'barkeit in jede Richtung folgt nicht totale Diff'barkeit, sobald die partiellen Ableitungen nicht stetig sind!

\subsection{Mehrdimensionale Kettenregel}
Sind $f:\mathbb{R}^n\to \mathbb{R}^l$ und $g:\mathbb{R}^l\to \mathbb{R}^m$ diff'bare Abbildungen, dann ist auch $g\circ f: \ \mathbb{R}^n\to \mathbb{R}^m$ diff'bar. Die Ableitung im Punkt p ist dann als die Hintereinanderausführung der Ableitung von \f in p und der Ableitung von g in $f(p)$ definiert
\begin{equation}
    D(g\circ f)_p = Dg_{f(p)}\circ Df_p
\end{equation}

\noindent Wird die Verkettung mit $h(x)= g\circ f$ definiert gilt dann 
\begin{equation}
    \frac{\partial h_i}{\partial x_j}(p) = \sum_{k=1}^{l} \frac{\partial g_i}{\partial y_k}(f(p)) \cdot \frac{\partial f_k}{\partial x_j}(p)
\end{equation}
hierbei werdem die Koordinaten im Def-Bereich von f als $x=(x_1,\dots,x_n)$ und die Koordinaten im Bildraum $\mathbb{R}^l$ von \f mit $y=(y_1,\dots,y_l)$ bezeichnet.

\subsection{Satz von Schwarz}
Existiert die partielle Ableitung ist $\frac{\partial^2 f}{\partial x_j \partial x_i}(x_0)$ die zweite partielle Ableitung von f nach $x_i$ und $x_j$ (in dieser Reihenfolge).\\

Ist f in einer Umgebung von $x_0 \in \mathbb{R}^n$ stetig, existieren in $x_0$ $f_{x_i},f_{x_j},f_{x_i,x_j},f_{x_j,x_i}$ und sind die zweiten partiellen Ableitungen stetig in $x_0$ so gilt
\begin{equation}
    \frac{\partial^2 f}{\partial x_i \partial x_j} = \frac{\partial^2 f}{\partial x_j \partial x_i}
\end{equation}
Das ist gleichbedeutend damit, dass die Hesse-Matrix symmetrisch ist.

\section{Lokale Extrema}

\newlinepar{Vorgehensweise: Extrema im $\mathbb{R}^n$ bestimmen}
\begin{enumerate}
    \item Gradient $\nabla f(x)$ bestimmen
    \item $\nabla f(x) = 0$ lösen
    \item Hesse-Matrix betrachten um Art des Extremums zu bestimmen  (\nameref{subsec:zweite-ableitung-krit-hessematrix})
\end{enumerate}

\subsection{Satz von Fermat}
\begin{itemize}
    \item $U \subset \mathbb{R}$ offen
    \item $f$ besitze in $x_0 \in U $ ein lokales Extremum und dort partiell diff'bar \\
    $\Rightarrow \nabla f(x_0) = 0$ 
    \item gilt $\nabla f(x_0) = 0$ heißt $x_0$ kritischer oder stationärer Punkt von $f$ 
\end{itemize}

\subsection{Zweite Ableitung Kriterium}
\label{subsec:zweite-ableitung-krit-hessematrix}
Es sei $f \varepsilon C^2(U)$ und $x_0 \in U$ ist kritischer Punkt $\Rightarrow \nabla f(x_0) = 0$
\begin{itemize}
    \item Hesse-Matrix positiv definit $\Rightarrow$ Minimum
    \item Hesse-Matrix negativ definit $\Rightarrow$ Maximum
    \item Hesse-Matrix indefinit $\Rightarrow$ Sattelpunkt
    \item Hesse-Matrix semi-positiv definit $\Rightarrow$ Minimum oder Sattelpunkt
    \item Hesse-Matrix semi-negativ definit $\Rightarrow$ Maxiumum oder Sattelpunkt
    \item Hesse-Matrix semi definit $\Rightarrow$ keine Aussage möglich. Betrachtung mit Definition und ggf. $\varepsilon$-Umgebung
\end{itemize}

\section{Implizite Funktionen}
\subsection{Lokale Injektivität}
$f \in C^2(U,\mathbb{R}^n$), $a \in U$ und $\vert Jf(a)\vert \neq 0$. Dann ist f lokal injektiv: \\
Es gibt offene Umgebung $U(a) \subset U$, so dass f auf diese Umgebung eingeschränkt injektiv ist.

\subsection{Lemma 2.5.5}
Es sei $f \in C^1(U,\mathbb{R}^n)$ injektiv, $a \in U$ mit $b := f(a)$. b ist dann innerer Punkt von $f(U)$, d.h. es existiert eine Umgebung $V(b)$ mit $V(b) \ \subset  \ f(U)$. \\
Mit anderen Worten: \f ist lokal bijektiv, so dass die Gleichung $f(x) \ = \ y$ für jedes $y \in V(b)$ genau eine Lösung $x \in U$ besitzt.

%\todo[inline]{einfügen von Lemma 2.5.5}

\subsection{Homöomorphismus und Umkehrfunktion}
Es sei $f \in C^1(U,\mathbb{R}^n)$ (global) injektiv, $Jf(x)\neq 0$ für alle $x \in U$. Dann ist $f(U)$ offen und $f: U \rightarrow f(U)$ ist ein Homöomorphismus (f ist bijektiv und stetig). Weiter ist $f^{-1}: f(U) \rightarrow U$ ebenfalls stetig.
\\m
Außerdem gilt unter den Vorraussetzungen $f \in C^1(U,\mathbb{R}^n)$, $a \in U$ mit $Jf(a)\neq 0$ und $b=f(a)$, dass es immer eine offene Umgebung $U(a)$ bzw. $V(b)$ gibt, auf welche eingeschränkt f ein Homöomorphismus ist.\\
$f: U(a) \rightarrow V(b)$

\subsection{Umkehrformel}
Es sei $f \in C^1(U,\mathbb{R}^n)$ injektiv mit $Jf(x)\neq 0$ für alle $x \in U$.\\
Dann ist $f^{-1}: f(U)\rightarrow U$ stetig diff'bar und $y=f(x) \in f(U)$ gilt die Umkehrformel:\\
$Df^{-1}(y) = \left(\frac{\partial f^{-1}_k}{\partial y_l}(y)\right)_{k,l=1}^{n} = (Df(x))^{-1} = {\big (}{\big(}\frac{\partial f_j}{\partial x_i}(x){\big )}_{i,j=1}^{n}{\big )}^{-1}$

\subsection{Stetig Diff'barkeit der Umkehrfunktion}
Existiert die Umkehrfunktion von \f und ist \f in einer Umgebung U p-mal stetig diff'bar, so ist auch die Umkehrfunktion $f^{-1}$ p-mal stetig diff'bar auf $f(U)$.

\subsection{Satz über inverse Abbildung}


%\todo[inline]{Satz über inverse Abbildungen: Skript 2.5.11}

\subsection{Satz über implizite Funktionen}
Informell liefert der Satz über implizite Funktionen ein Funktion, deren Bild eine Menge von Punkten in der Definitionsmenge beschreibt, welche auf gleichem Potenzial liegen.
\begin{equation*}
    {\frac {\partial F}{\partial x}}(x,y(x))+{\frac {\partial F}{\partial y}}(x,y(x))\cdot {\frac {\partial y}{\partial x}}(x)=0
\end{equation*}
wobei dann nach ${\frac {\partial y}{\partial x}}$ umgestellt:
\begin{equation*}
    {\frac {\partial y}{\partial x}}(x)=-\left({\frac {\partial F}{\partial y}}(x,y(x))\right)^{-1}\cdot {\frac {\partial F}{\partial x}}{\big (}x,y(x){\big )}
\end{equation*}
Der Satz lässt sich so auch auf (nicht notwendigerweise lineare) Gleichungssysteme mit n Gleichungen anwenden. \\
Wobei ${\frac {\partial y}{\partial x}}$ und ${\frac {\partial F}{\partial x}}{\big (}x,y(x)_1...y(x)_k{\big )}$ dann jeweils Vektoren der Länge n sind und 
\begin{equation*}
    \left({\frac {\partial F}{\partial y}}(x,y(x)_1...y(x)_k)\right)^{-1}={\big (}J_f{\big )}^{-1} 
\end{equation*}
gerade die invertierte Jacobi-Matrix über ${\big (}y_1...y_k{\big )}$
 nach denen aufgelöst werden soll ist. \\
 Alternativ zur Berechnung der Inversen und ggf. einfacher können die Gleichungen auch für sich abgeleitet werden. Umstellen und Einsetzten führt dann zum gleichen Ergebnis.

%\todo[inline]{letzten Abschnitt bitte überprüfen!}

%\todo[inline]{Implizite Differentiation mit Def aus Skript:2.5.18 ergänzen}


\subsection{Satz über implizite Funktionen im \texorpdfstring{$\mathbb{R}^2$}{R2}}
\begin{itemize}
    \item $U \ \subset \ \mathbb{R}^2$ offen
    \item $f: U\rightarrow \mathbb{R}$ in $C^1(U)$ und $\begin{pmatrix} a \\b  \end{pmatrix} \in U$
    \item $\nabla f(a,b)\neq 0$ und $c:=f(a,b)$
\end{itemize}
Dann gibt es eine offene Umgebung $U(a,b) \ \subset \ U$, ein $\epsilon > 0$ und eine $C^1$-Kurve $\gamma: (-\epsilon,\epsilon) \rightarrow U(a,b)$ mit $\gamma(0) = \begin{pmatrix} a \\b  \end{pmatrix}$.\\
Dann gibt $\gamma$ gerade die Funktion der gesuchten Niveaumenge an.

\subsection{Lagrange Multiplikatorenregel}
\begin{itemize}
    \item $M \ \subset \ \mathbb{R}^n$ offen
    \item $f: M \rightarrow \mathbb{R}$ und $f \in C^1(M)$
    \item $g: M \rightarrow \mathbb{R}^s$ und $g \in C^1(M)$
    \item $s<n$
    \item Ableitung von g hat maximalen Rang im Punkt $x_0$
\end{itemize}
Wenn $x_0$ lokales Extremum von f bzgl. der Nebenbedingung ist, gilt\\
$\exists \lambda \in \mathbb{R}^s: \frac{d}{dx}(f(x)+\lambda^T g(x)) = 0$\\
Das ist äquivalent zu \\
$\nabla f(x) = \lambda \nabla g(x)$. Hieraus folgt insbesondere, dass die beiden Gradienten parallel sind!\\
Sind mehrere Nebenbedingungen g gegeben gilt
\begin{equation*}
   \nabla f(x) = \sum_{k=1}^{n} \lambda_k \nabla g(x)
\end{equation*}


\newlinepar{Vorgehensweise: Extrema unter Nebenbedingungen bestimmen}
%\todo[inline]{proofread, müsste wohl passen -jonas}
\begin{enumerate}
    \item Gradienten $\nabla f(x)$ und $\nabla g(x)$ bilden
    \item $\nabla f(x) = \lambda \nabla g(x)$ aufstellen
    \item Resultierendes Gleichungssystem für Komponentenfunktionen lösen, wenn nötig mit bekannten Verfahren für das Lösen von LGS (Gauß).
    \item Prüfen, ob/welche Lösung Anfangsbedingungen erfüllt
\end{enumerate}
Achtung: Einfache \emph{Hesse-Matrix bringt keine Aussage} über die Art des Extremums unter Nebenbedingungen, sondern nur im allgemeinen Fall (ohne NB). (Geränderte Hesse-Matrix bringt Aussage, ist aber sehr aufwendig (meiner Meinung nach nicht machbar in Klausur))\\
\noindent
Bemerkung: Ist der zulässige Bereich kompakt und die Funktion stetig, \emph{nimmt sie Maximum und Minimum an}.\\
\noindent
Es kann aber die Hesse-Matrix der Lagrange-Funktion untersucht werden, bspw. $F(x,y,z)=f(x,y,z)-\lambda g(x,y,z)$. Diese muss dann wie gewohnt auf Definitheit geprüft werden.