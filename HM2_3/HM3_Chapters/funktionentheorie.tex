%\mathbb{set}$ for number sets


\section{Wdh: Grundlagen}
\subsubsection{Definition: Argument}
Es ist $z\in\mathbb{C}\setminus\{0\}$. Jede Zahl $\phi\in\mathbb{R}$ mit $e^{i\phi}=\frac{z}{|z|}$ heißt \textit{Argument} von $z$ und wird mit $arg(z)$ bezeichnet. Das eindeutig bestimmte Argument auf dem Intervall $(-\pi,\pi]$ heißt \textit{Hauptwert des Arguments} und wird mit $Arg(z)$ bezeichnet. 

\subsubsection{Zusammenfassung von Lemmata/Defs}
\begin{enumerate}
    \item Die Funktion $f: \mathbb{C}\setminus(-\infty,0]\to\mathbb{R} \; f(z)=Arg(z)$ ist stetig
    \item Für jedes $z\in\mathbb{C}\setminus\{0\}$ hat die Gleichung $\omega^2=z$ genau zwei Lösungen $w=\pm\sqrt{|z|}e^{\frac{iArg(z)}{2}}$
    \item $g:\mathbb{C}\setminus(-\infty,0]\to\mathbb{C} \; g(z)=\sqrt{|z|}e^{\frac{iArgs(z)}{2}}$ ist die Wurzelfunktion bzw. der Hauptwert der Wurzel
    \item für $z\in\mathbb{C}\setminus(-\infty,0]$ heißt $Log(z)=log|z|+iArg(z)$ der Hauptwert des Logarithmus von $z$
\end{enumerate}

\section{Komplexe Differenziation}
\label{sec:komplexe-differentiation}
\subsection{Formale Definition}
Es ist $M\subset\mathbb{C}$ offen und nicht leer und $f:M\to\mathbb{C}$ eine Funktion. $f$ heißt komplex diff'bar in $z_0\in M$, wenn der Grenzwert 
\begin{equation*}
    \frac{\diff}{\diff z}f(z_0) = \lim_{z\to z_0}\frac{f(z)-f(z_0)}{z-z_0}
\end{equation*}
existiert. Ist $f$ in jedem Punkt von $M$ diff'bar heißt $f'$ die Ableitung von f.

\paragraph{Bemerkungen}
\begin{enumerate}
    \item Rechenregeln für die Ableitungen aus dem reellen können auch im komplexen angewendet werden
    \item $f:M\to\mathbb{C}$ ist genau dann diff'bar in $z_0$ wenn eine Darstellung der Form 
            \begin{equation*}
                f(z)=f(z_0)+c(z-z_0)+\phi(z,z_0)
            \end{equation*}
            existiert, mit einem $c\in\mathbb{C}$ und einem Restterm $\phi(z,z_0)$ der schneller als linear abfällt.
    \item $\mathbb{C}$ kann durch Zuordnung von $z=x+iy\to \left(\begin{smallmatrix}x \\ y\end{smallmatrix}\right)$ als $\mathbb{R}^2$ angesehen werden        
\end{enumerate}

\subsubsection{Äquivalente Aussagen}
Für eine Funktion $f:D\subset\mathbb{C}\to\mathbb{C}$ sind äquivalent
\begin{enumerate}
    \item $f$ ist in $z_0$ komplex diff'bar
    \item $f$ ist in $z_0$ reell diff'bar und $f_{\bar{z}}(z_0)=0$
    \item $f$ ist in $z_0=x_0+iy_0$ reell diff'bar und $u=\operatorname{Re}(f)$ und $v=\operatorname{Im}(f)$ erfüllen die CR-DGL
    \begin{equation}
        \tag{Cauchy Riemann DGL}
        \label{eq:cr-dgl}
        \begin{gathered}
                \frac{\diff}{\diff x}u(x_0,y_0) = \frac{\diff}{\diff y}v(x_0,y_0)\\
                \frac{\diff}{\diff y}u(x_0,y_0) = - \frac{\diff}{\diff x}v(x_0,y_0)
        \end{gathered}
    \end{equation}
            
\end{enumerate}

\section{Holomorphie}
\subsection{Definition: Holomorphie}
Es ist $M\subset\mathbb{C}$ nicht leer und offen, $z_0\in M $ und $f:M\to\mathbb{C}$ eine Funktion
\begin{enumerate}
    \item $f$ heißt \textit{holomorph} oder \textit{analytisch} in $z_0$, wenn $f$ in einer Umgebung $U_{\varepsilon}(z_0)$ diff'bar (\ref{sec:komplexe-differentiation}, z.b. \eqref{eq:cr-dgl}) ist 
    \item $f$ heißt \textit{holomorph} oder \textit{analytisch} auf $M$, wenn $f$ in jedem Punkt von $M$ holomorph ist
\end{enumerate}

\noindent Linearkombinationen, Produkte, Verkettungen und Quotienten von holomorphen Funktionen sind auf ihrem Definitionsgebiet auch holomorph.


\subsection{Möbius-Transformation}
Es ist durch $A=\left(\begin{smallmatrix}a & b \\ c & d \end{smallmatrix}\right)$ mit $a,b,c,d\in\mathbb{C}$ und $det(A)\neq 0$ eine Möbiustransformation gegeben. Die Trafo ist als folgende Abbildung definiert
\begin{equation*}
    \omega:\mathbb{C}_{\infty}\to\mathbb{C}_{\infty}\; \text{mit} \; \omega(z)=\frac{az+b}{cz+d}
\end{equation*}
Weiter ist definiert, dass $\omega(z)=\infty$ für $z=-\frac{d}{c}$ und $\omega(\infty)=\frac{a}{c}$.
Möbius-Trafos sind bijektive Abbildungen von $\mathbb{C}_{\infty}$ auf sich selbst. Außerdem bildet die Menge der Möbius-Trafos bzgl. der Hintereinanderausführung eine Gruppe.

\paragraph{Bemerkungen:}
\begin{itemize}
    \item für invertierbare Matrizen $A,B\in\mathbb{C}^{2x2}$ gilt $\omega_A \circ \omega_B=\omega_{AB}$ \\und $\omega_{A^{-1}} = \omega_A^{-1}$ 
    \item Jede Möbius-Trafo lässt sich als Hintereinanderausführung von Drehstreckungen, Verschiebungen und einer Inversion schreiben
    \item Jede Trafo hat zwei Fixpunkte, die auch zusammenfallen können (Ausnahme die Identität)
\end{itemize}



\subsubsection{Doppelverhältnis zur Berechnung der Möbius-Trafo}
Es ist durch
\begin{equation*}
    \operatorname{DV}(z_1,z_2;z_3,z_4):=\frac{z_2-z_4}{z_2-z_3}\cdot\frac{z_1-z_3}{z_1-z_4}
\end{equation*}
,mit $(z_1,z_2,z_3,z_4)\subset\mathbb{C}$ das Doppelverhältnis definiert.\\
Gilt $\{z_1,z_2,z_3,z_4\}\subset\mathbb{C}$ mit $|\{z_1,\dots,z_4\}|\geq 3$ und $\phi$ ist eine Möbiustrafo
\begin{equation*}
    \operatorname{DV}(\phi(z_1),\phi(z_2);\phi(z_3),\phi(z_4))=\operatorname{DV}(z_1,z_2;z_3,z_4)
\end{equation*}
Damit kann dann die Möbiustrafo berechent werden.
\paragraph{Beispiel}
Bestimmen einer Möbiustrafo für welche gilt $T(0)=-1$, $T(i)=-2+i$ und $T(-i)=-2-i$. Dann ist mit obiger Gleichung die Möbistrafo durch das Lösen der folgenden Gleichung zu bestimmen
\begin{equation*}
    \frac{T(z)+2-i}{T(z)+1}\cdot\frac{-2-i+1}{-2-i+2-i}=\frac{z-i}{z}\cdot\frac{-i}{-i-i}
\end{equation*}

\subsection{Ganze Funktionen}
Eine auf ganz $\mathbb{C}$ holomorphe Funktion heißt auch \textit{ganze Funktion}. Ist die Funktion kein Polynom heißt die Funktion \textit{transzendente Funktion}.\\
Der Satz von Liouville besagt: eine beschränkte ganze Funktion ist konstant.


\section{Komplexe Integralrechnung}

\subsection{Komplexes Kurvenintegral}
Ist $M\subset\mathbb{C}$, $f:M\to\mathbb{C}$ eine stetige Funktion. $\gamma:[a.b]\to\mathbb{C}$ eine stückweise stetig diff'bare Parametrisierung der Kurve, dann ist
\begin{equation*}
    \int_{\gamma}f(z)\diff z = \int_a^b f(\gamma(t))\cdot \gamma'(t)\diff t
\end{equation*}
das komplexe Kurvenintegral von $f$ über die Kurve $\gamma$.

\paragraph{Bemerkung}
Da $\mathbb{C}$ auch als $\mathbb{R}^2$ angesehen werden kann, können komplexe Kurvenintegrale auch in reelle umgewandelt werden. Es ist $f(z)=u(z)+iv(z)$ mit $z(t)=x(t)+iy(t)$, dann erhält man
\begin{equation*}
    \int_{\gamma} f(z)\diff z = \int_{\gamma}(u \diff x - v\diff y) + i \int_{\gamma}(v\diff x + u\diff y)
\end{equation*}
zwei reelle Kurvenintegrale.

\subsection{Nützliche Lemmata und Bemerkungen zu komplexen Kurvenintegralen}
Es ist $M\subset\mathbb{C}$ und $f:M\to\mathbb{C}$ eine stetige Funktion und $\gamma$ die Parametrisierung einer stetig diff'baren Kurve
\begin{itemize}
    \item Ist $|f(z)|\leq c$ dann gilt $|\int_{\gamma}f(z)\diff z|\leq c\cdot L(\gamma)$
    \item hat $f(z)$ eine Stammfunktion $F(z)$ so gilt $\int_\gamma f(z)\diff z = F(\gamma(b)) - F(\gamma(a))$, wobei $a,b$ gerade Anfangs- und Endpunkt der Parametrisierung sind
\end{itemize}

Es ist $G\subset\mathbb{C}$ ein Gebiet und $f:G\to\mathbb{C}$ eine stetige Funktion, dann sind die folenden Aussagen äquivalent
\begin{itemize}
    \item $f$ hat auf $G$ eine Stammfunktion
    \item Für jede stückweise stetig diff'bare geschlossene Kurve mit Parametrisierung $\gamma$ gilt $\int_\gamma f(z)\diff z = 0$
    \item $\int_\gamma f(z)\diff z$ ist wegunabhängig
\end{itemize}


\subsection{Cauchyscher Integralsatz}
\newlinepar{Voraussetzungen} 
$G \in \mathbb{C}$ sternförmiges Gebiet, $p \in G$, $f: G \rightarrow \mathbb{C}$ auf $G$ stetig und auf $G \setminus \{p\}$ holomorph.

\paragraph{Aussagen}
\begin{itemize}
    \item $f$ besitzt auf $G$ eine Stammfunktion.
    \item Integral über stückweise stetig diffbare geschlossene Kurve $=0$:
\end{itemize}

\begin{equation}
    \int_\gamma f(z) \diff z = 0
\end{equation}

\newlinepar{Beispiel}
%\todo[inline]{Beispiel zu Cauchy Integralsatz}


%\todo[inline]{Bemerkungen aus \glqq Ausflug Homotopie 6.1.6\grqq erwähnen?}

\subsection{Cauchysche Integralformel}
\label{Cauchy_if}
\newlinepar{Voraussetzungen}
$G \in \mathbb{C}$ sternförmiges Gebiet, $f: G \rightarrow \mathbb{C}$ auf $G$ holomorph.\\
$\gamma$ stückweise stetig diffbare geschlossene Kurve in $G$.\\
$z_0 \in G$ liegt nicht auf $\gamma$.

\paragraph{Aussagen}
\begin{equation}
    f(z_0) N_\gamma(z_0) = \frac{1}{2\pi i} \int_\gamma \frac{f(z)}{z-z_0}\diff z
\end{equation}


\newlinepar{Beispiel}
%\todo[inline]{Beispiel zu Cauchy Integralformel}
\begin{equation*}
    \int_{|z-3i|=2}\frac{1}{z^2+9}\diff z =  \int_{|z-3i|=2} \frac{\frac{1}{z+3i}}{z-3i}=2\pi i \frac{1}{3i+3i}=\frac{\pi}{3}
\end{equation*}
Es muss darauf geachtet werden, dass evlt. weitere Problemstellen außerhalb liegen.

%\todo[inline]{einfügen von 6.1.18}

\section{Singularitäten}

\subsection{Definition}
Es sei $f:\dot{U}_r(z_0) \to \mathbb{C}$ holomorph mit $r>0$ und $z_0\in \mathbb{C}$, dann besitzt $f$ in $z_0$ eine isolierte Singularität.

\subsection{Klassifizerung}
\subsubsection{Klassen von Singularitäten}
Eine Singularität heißt,
\begin{itemize}
    \item \textit{hebbar}, wenn $f$ in einer punktierten Umgebung von $z_0$ beschränkt ist
    \item \textit{Polstelle}, wenn $lim_{z\to z_0}|f(z)|=+\infty$
    \item \textit{wesentliche Singularität}, wenn weder hebbar noch Pol in $z_0$
\end{itemize}

\subsubsection{Riemann'scher Hebbarkeitssatz}
Es ist $G\subset\mathbb{C}$ ein Gebiet und $f\in H(G)$, $z_0$ ist eine isolierte Singularität von $f$. Es handelt sich genau dann um eine hebbare Singularität, wenn eine in $z_0$ holomorphe Funktion $g$ existiert, so dass $f(z) = g(z)$ für alle $z\in \dot{U}_r(z_0)$ mit $r>0$.

\subsection{Laurent-Reihen}
\subsubsection{Definition: Laurent-Reihe}
$f$ ist eine auf dem Kreisring  $G=\{z\in\mathbb{C} | R_1 < |z-z_0| < R_2\}$ holomorphe Funktion, dann hat $f$ eine Darstellung als Laurent-Reihe der Form
\begin{equation*}
    f(z)=\sum_{k=-\infty}^{+\infty}a_k(z-z_0)^k
\end{equation*}
wenn man mit dem inneren Radius $R_1$ des Kreisrings gegen null geht und $R=R_2$ setzt, erhält man gerade den Fall einer punktierten Umgebung $\dot{U}_R(z_0)$.\\
Die Koeffizienten können theoretisch folgendermaßen berechnet werden
\begin{equation*}
    a_k=\frac{1}{2\pi i }\int_{|z-z_0|=r}\frac{f(z)}{(z-z_0)^{k+1}}\diff z
\end{equation*}
Die Reihe konvergiert auf dem Kreisring \textit{absolut} gegen f und auf jeder kompakte Teilmenge \textit{gleichmäßig}. Die Darstellung ist eindeutig.

\paragraph{Bemerkung:}
\begin{itemize}
    \item $\sum_{k=-\infty}^{-1}a_k(z-z_0)^k$ heißt \textit{Hauptteil} der Laurent-Reihe
    \item $\sum_{k=0}^{+\infty}a_k(z-z_0)^k$ heißt \textit{Nebenteil} der Laurent-Reihe
\end{itemize}

\subsubsection{Klassifizierung von isolierten Singularitäten anhand des Hauptteils}
\begin{itemize}
    \item $z_0$ ist genau dann eine hebbare Singularität, wenn $a_k=0$ für alle $k\in\mathbb{Z}\setminus\mathbb{N}_0$ (kein Hauptteil)
    \item $z_0$ ist genau dann eine Polstelle der Ordnung n, wenn $a_{-n}\neq 0 $ und $a_k=0$ für alle für alle $k\in\mathbb{Z}$ mit $k<-n$, d.h. heißt der Hauptteil hat eine endliche Anzahl an Gliedern
    \item $z_0$ ist eine wesentliche Singularität, wenn der Hauptteil unendliche viele Glieder hat
\end{itemize}

\section{Residuensatz}
\subsection{Residuen}
Mit dem Residuensatz können Kurvenintegrale über geschlossene Kurven gelöst werden. Falls $G$ ein homotop einfaches zusammenhängendes Gebiet ist und $f$, eine Funkton, auf diesem Gebiet holomorph.

\subsubsection{Definition: Residuum}
Es ist $G$ wie oben und $f$ eine auf $G\setminus\{z_1,\dots,z_m\}$ holomorphe Funktion. Dann ist das Residuum von $f$ im Punkt $z_0$ als
\begin{equation*}
    Res_{z_o}(f)= \frac{1}{2\pi i}\int_{|z-z_0|=r}f(z)\diff z    
\end{equation*}
definiert. Wobei die Umgebung so gewählt wurde, dass sie nur eine Singularität enthält.\\
\noindent Direkte Folgerungen davon sind
\begin{itemize}
    \item ist $f$ holomorph in $z_0$, so ist $Res_{z_0}(f)=0$
    \item ist $f(z)=\sum_{k=-\infty}^{+\infty}a_k (z-z_0)^k$ eine Laurent-Reihendarstellung von $f$, dann ist $Res_{z_0}(f) = a_{-1}$
\end{itemize}

\subsubsection{Bestimmung von Residuen an Polstellen}
Besitzt $f(z)$ in $z_0$ einen Pol der Ordnung $n$, dann folgt aus der Cauchyschen Integralformel
\begin{equation*}
    \res_{z_0}(f)=\frac{1}{(n-1)!}\left.\left(\left(z-z_0\right)^nf(z)\right)^{(n-1)}\right|_{z=z_0}
\end{equation*}

\subsection{Residuensatz}
Es ist $G\subset \mathbb{C}$ ein homotop einfach zusammenhängendes Gebiet. Eine Funktion $f$, die auf $G\setminus\{z_1,\dots,z_m\}$ holomorph ist. Dann gilt für jede stückweise stetig diff'bare Kurve $\gamma$ in $G\setminus\{z_1,\dots,z_m\}$ gilt

\begin{equation*}
    \int_{\gamma}f(z)\diff z = 2\pi i \sum_{k=1}^{m}Res_{z_k}(f) \cdot N_{\gamma}(z_k)
\end{equation*}

Die Aussage bleibt wahr für beliebig viele isolierte Singularitäten, solange eventuelle Häufungspunkte auf dem Rand von $G$ liegen.\\
\noindent Residuensatz ist eine verallgemeinerung der Cauchy-Integralformel (\ref{Cauchy_if}).


\subsubsection{Bestimmen von Residuen}
\begin{itemize}
    \item Residuum ist linear
    \item $f$ hat in $z_0$ einen Pol \textit{erster Ordnung}, dann kann das Residuum über $Res_{z_0}(f) = \lim_{z \to z_0}(z-z_0)f(z)$ bestimmt werden 
    \item $f$ hat in $z_0$ einen Pol \textit{erster Ordnung} und $g$ ist holomorph, dann kann das Residuum über $Res_{z_0}(f\cdot g) = g(z_0)\cdot Res_{z_0}(f)$ bestimmt werden
    \item $f$ hat in $z_0$ einen Pol \textit{n-ter Ordnung}, dann kann das Residuum über $Res_{z_0}(f) = \frac{1}{(n-1)!}\frac{\diff^{n-1}}{\diff z^{n-1}}[(z-z_0)^nf(z)]$ ausgewertet an $z=z_0$, bestimmt werden
    \item $f,g \in H(U_R(z_0))$ mit $f(z_0)\neq 0$, $g(z_0)=0$ und $g'(z_0)\neq 0$, dann hat $\frac{f}{g}$ in $z_0$ einen Pol erster Ordnung und es gilt $Res_{z_0}(\frac{f}{g})=\frac{f(z_0)}{g'(z_0)}$
\end{itemize}

\subsubsection{Bestimmen reeller Integrale mit dem Residuensatz}
Es ist $G$ ein homotop einfach zusammenhängendes Gebiet. $f$, eine Funktion, ist auf $G\setminus\{z_1,\dots,z_m\}$ holomorph. $\{z_1,\dots,z_m\}$ sind verschiedene Punkte der offenen oberen komplexen Halbebene: $H = \{z\in \mathbb{C} | Im(z) > 0\}$ und $H \subset G$.\\
Außerdem gelte für jede Folge $(\omega)_{n \in \mathbb{N}} \subset H$ mit $\lim_{n \to \infty}|\omega_n|=+\infty$, dass $\lim_{n \to \infty}f(\omega_n)\omega_n=0$. Dann konvergiert das Integral, so dass gilt
\begin{equation*}
    \int_{-\infty}^{+\infty}f(x)\diff x = 2\pi i \sum_{k=1}^m Res_{z_k}(f)
\end{equation*}

Es ist zu beachten, dass das Integral auch tatsächlich konvergiert.\\
\noindent Auf Blatt 14 wird außerdem noch das Jordan'sche Lemma gezeigt, welches besagt dass für $H$ und $f$ wie oben, mit der Abschwächung, dass $zf(z)$ beschränkt sein muss, gilt (für $\omega>0$)
\begin{equation*}
    \int_{-\infty}^{+\infty} e^{iwx}f(x)\diff x = 2\pi i\sum_{k=1}^m Res_{z_k}(e^{iwz}f(z))
\end{equation*}

\paragraph{Bemerkung}
Es wurde außerdem gezeigt, dass dies insbesondere für eine rationale Funktion, mit größerem Nenner- als Zählergrad gilt.\\
\noindent Es ist zu beachten, dass sich die Aussage für $w<0$ umdreht, so dass Punkte in der unteren komplexen Halbebene betrachtet werden. Je nach Kurve ist dann auch die Richtung, in welcher die Parametrisierung durchlaufen wird relevant und muss unter Umständen geändert werden.\\
Bspw. (A42 (d)) $cos(\omega x) = \frac{e^{i\omega x} + e^{-i\omega x}}{2}$ muss der eine Teile für die obere und der andere für die untere Halbebene betrachtet werden um das gesuchte Ergebnis zu erhalten. (alternativ mit $cos(\omega x) = \operatorname{Re}(e^{i\omega x})$)

\subsubsection{Bestimmen von Reihen mit Hilfe des Residuensatzes}

%\todo[inline]{Bitte prüfen}

Es wird für eine konvergente Reihe $\sum_{k=1}^{+\infty}f(k)$ ein Ergebnis gesucht. Man wählt eine passende Funktion oft $g(z)=\pi cot(\pi z)$, da diese Polstellen bei $z\in \mathbb{Z}$ hat. Nach dem Residuensatz gilt dann für das Kurvenintegral
\begin{equation*}
    \oint_{\gamma}f(z)\pi cot(\pi z)\diff z = 2\pi i \sum_{k=1}^m Res_{z_k}(f(z)\pi cot(\pi z))
\end{equation*}
Da Residuen aber linear sind, können diese in Residuen von $f$ und Residuen von $g$ unterteilt werden

\begin{equation*}
    \oint_{\gamma}f(z)\pi cot(\pi z)\diff z = 2\pi i(\sum Res_f(f\cdot g) + \sum Res_g(f\cdot g))
\end{equation*}

Es wird nun gezeigt, dass durch ausweiten der Kurve (bspw. bei Kreis $R\to \infty$) das Integral auf der linken Seite gegen 0 geht, dann folgt
\begin{equation*}
    \sum Res_g(f\cdot g) = - \sum Res_f(f\cdot g)
\end{equation*}

da durch ausweiten der Kurve gegen unendlich nun die gesamte reelle Achse enthalten ist und der cotangens jeweils bei ganzzahligen Werten Singularitäten besitzt folgt

\begin{equation*}
    \sum_{k=-\infty}^{\infty}f(k) = - \sum Res_f(f\cdot g)
\end{equation*}

Durch Bestimmen der Residuen von $f$ und umstellen, kann das Ergebnis der Reihe bestimmt werden.

\subsubsection{Anwendung des Residuensatz auf Trig. Fkt.}
Es sind P und Q Polynome in zwei Veränderlichen, mit $Q(x,y)\neq 0$ für alle $x,y\in\mathbb{R}$ mit $x^2+y^2=1$. Es wird die Funktion $f$ definiert als 
\begin{equation*}
    f(z)=\frac{1}{zi}\frac{P(\frac{1}{2}(z+\frac{1}{z}),\frac{1}{2i}(z-\frac{1}{z}))}{Q(\frac{1}{2}(z+\frac{1}{z}),\frac{1}{2i}(z-\frac{1}{z}))}
\end{equation*}

dann gilt
\begin{equation*}
    \int_{0}^{2\pi} \frac{P(\cos{t},\sin{t})}{Q(\cos{t},\sin{t})}\diff t = 2\pi i\sum_{|a|<1}\operatorname{Res}_a(f)
\end{equation*}

Es wird also die Summe der Residuen aller Singularitäten von f gebildet, die innerhalb des Einheitskreises liegen.

\paragraph{Beispiel}
Mit der oben beschriebenen Methode können also Integrale, welche trig. Fkt. enthalten, auf die Summe der Residuen einer rationalen Fkt. reduziert werden.\\
Es ist das Integral 
\begin{equation*}
    \int_0^\pi \frac{\cos{t}^2}{1+i\sin{t}\cos{t}} \diff t
\end{equation*}

\noindent gegeben. Man erhält dann also für die oben definierte Funktion $f$
\begin{equation*}
    f(z)=\frac{1}{iz}\frac{\frac{1}{4}(z+\frac{1}{z})^2}{1+\frac{1}{4}(z-\frac{1}{z})(z+\frac{1}{z})}
    =\frac{1}{iz}\frac{z^4+2z^2+1}{z^4+4^2-1}
\end{equation*}

$f$ hat Singularitäten in $z=0$ und $z=\pm \sqrt{-2\pm \sqrt{5}}$. Es werden nun die Residuen der Singularitäten berechent, welche innerhalb des Einheitskreises liegen.


\begin{align*}
    \operatorname{Res}_0(f)&=-i\frac{z^4+2z^2+1}{5z^4+12z^2-1}\big|_{z=0}=i\\
    \operatorname{Res}_{\pm \sqrt{-2+\sqrt{5}}}(f)&=-i\frac{z^4+2z^2+1}{5z^4+12z^2-1}\big|_{z=\pm \sqrt{-2+\sqrt{5}}}=-i\frac{3-\sqrt{5}}{10-4\sqrt{5}}
\end{align*}


Damit erhält man durch nutzen der Symmetrieeigenschaften der Funktion

\begin{align*}
    \int_0^\pi\frac{\cos{t}^2}{1+i\cos{t}\sin{t}}\diff t = 2\pi i\frac{i}{2}(1-\frac{3-\sqrt{5}}{5-2\sqrt{5}})=\frac{\pi}{\sqrt{5}}
\end{align*}
